\startchapter[title={Theory And Practice}]

\startsection[title={Troubleshooting}]

Perhaps the most valuable but difficult-to-learn skill any technical
person could have is the ability to troubleshoot a system. For those
unfamiliar with the term, {\em troubleshooting} means the act of
pinpointing and correcting problems in any kind of system. For an auto
mechanic, this means determining and fixing problems in cars based on
the car's behavior. For a doctor, this means correctly diagnosing a
patient's malady and prescribing a cure. For a business expert, this
means identifying the source(s) of inefficiency in a corporation and
recommending corrective measures.

Troubleshooters must be able to determine the cause or causes of a
problem simply by examining its effects. Rarely does the source of a
problem directly present itself for all to see. Cause/effect
relationships are often complex, even for seemingly simple systems, and
often the proficient troubleshooter is regarded by others as something
of a miracle-worker for their ability to quickly discern the root cause
of a problem. While some people are gifted with a natural talent for
troubleshooting, it is a skill that can be learned like any other.

Sometimes the system to be analyzed is in so bad a state of affairs that
there is no hope of ever getting it working again. When investigators
sift through the wreckage of a crashed airplane, or when a doctor
performs an autopsy, they must do their best to determine the cause of
massive failure after the fact. Fortunately, the task of the
troubleshooter is usually not this grim. Typically, a misbehaving system
is still functioning to some degree and may be stimulated and adjusted
by the troubleshooter as part of the diagnostic procedure. In this
sense, troubleshooting is a lot like scientific method: determining
cause/effect relationships by means of live experimentation.

Like science, troubleshooting is a mixture of standard procedure and
personal creativity. There are certain procedures employed as tools to
discern cause(s) from effects, but they are impotent if not coupled with
a creative and inquisitive mind. In the course of troubleshooting, the
troubleshooter may have to invent their own specific technique ---
adapted to the particular system they're working on --- and/or modify
tools to perform a special task. Creativity is necessary in examining a
problem from different perspectives: learning to ask different questions
when the \quotation{standard} questions don't lead to fruitful answers.

If there is one personality trait I've seen positively associated with
excellent troubleshooting more than any other, its technical curiosity.
People fascinated by learning how things work, and who aren't
discouraged by a challenging problem, tend to be better at
troubleshooting than others. Richard Feynman, the late physicist who
taught at Caltech for many years, illustrates to me the ultimate
troubleshooting personality. Reading any of his (auto)biographical books
is both educating and entertaining, and I recommend them to anyone
seeking to develop their own scientific reasoning/troubleshooting
skills.

\stopsection

\startsection[title={Questions to Ask Before
Proceeding},reference={sec:xtocid15790971}]

\startitemize[packed]
\item
  Has the system ever worked before? If yes, has anything happened to it
  since then that could cause the problem?
\item
  Has this system proven itself to be prone to certain types of failure?
\item
  How urgent is the need for repair?
\item
  What are the {\em safety concerns}, before I start troubleshooting?
\item
  What are the process quality concerns, before I start troubleshooting
  (what can I do without causing interruptions in production)?
\stopitemize

These preliminary questions are not trivial. Indeed, they are essential
to expedient and safe troubleshooting. They are especially important
when the system to be trouble-shot is large, dangerous, and/or
expensive.

Sometimes the troubleshooter will be required to work on a system that
is still in full operation (perhaps the ultimate example of this is a
doctor diagnosing a live patient). Once the cause or causes are
determined to a high degree of certainty, there is the step of
corrective action. Correcting a system fault without significantly
interrupting the operation of the system can be very challenging, and it
deserves thorough planning.

When there is high risk involved in taking corrective action, such as is
the case with performing surgery on a patient or making repairs to an
operating process in a chemical plant, it is essential for the worker(s)
to plan ahead for possible trouble. One question to ask before
proceeding with repairs is, \quotation{how and at what point(s) can I
abort the repairs if something goes wrong?} In risky situations, it is
vital to have planned \quotation{escape routes} in your corrective
action, just in case things do not go as planned. A surgeon operating on
a patient knows if there are any \quotation{points of no return} in such
a procedure, and stops to re-check the patient before proceeding past
those points. He or she also knows how to \quotation{back out} of a
surgical procedure at those points if needed.

\stopsection

\startsection[title={General Troubleshooting
Tips},reference={sec:xtocid15790972}]

When first approaching a failed or otherwise misbehaving system, the new
troubleshooter often doesn't know where to begin. The following
strategies are not exhaustive by any means, but provide the
troubleshooter with a simple checklist of questions to ask in order to
start isolating the problem.

As tips, these troubleshooting suggestions are not comprehensive
procedures: they serve as starting points only for the troubleshooting
process. An essential part of expedient troubleshooting is probability
assessment, and these tips help the troubleshooter determine which
possible points of failure are more or less likely than others. Final
isolation of the system failure is usually determined through more
specific techniques (outlined in the next section --- {\em Specific
Troubleshooting Techniques}).

\startsubsection[title={Prior
Occurrence},reference={sec:xtocid15790973}]

If this device or process has been historically known to fail in a
certain particular way, and the conditions leading to this common
failure have not changed, check for this \quotation{way} first. A
corollary to this troubleshooting tip is the directive to keep detailed
records of failure. Ideally, a computer-based failure log is optimal, so
that failures may be referenced by and correlated to a number of factors
such as time, date, and environmental conditions.

{\bf Example:} {\em The car's engine is overheating. The last two times
this happened, the cause was low coolant level in the radiator.}

{\bf What to do:} Check the coolant level first. Of course, past history
by no means guarantees the present symptoms are caused by the same
problem, but since this is more likely, it makes sense to check this
first.

If, however, the cause of routine failure in a system has been corrected
(i.e.~the leak causing low coolant level in the past has been repaired),
then this may not be a probable cause of trouble this time.

\stopsubsection

\startsubsection[title={Recent
Alterations},reference={sec:xtocid15790974}]

If a system has been having problems immediately after some kind of
maintenance or other change, the problems might be linked to those
changes.

{\bf Example:} {\em The mechanic recently tuned my car's engine, and now
I hear a rattling noise that I didn't hear before I took the car in for
repair.}

{\bf What to do:} Check for something that may have been left loose by
the mechanic after his or her tune-up work.

\stopsubsection

\startsubsection[title={Function
vs.~Non-Function},reference={xtocid15790975}]

If a system isn't producing the desired end result, look for what it
{\em is} doing correctly; in other words, identify where the problem is
{\em not}, and focus your efforts elsewhere. Whatever components or
subsystems necessary for the properly working parts to function are
probably okay. The degree of fault can often tell you what part of it is
to blame.

{\bf Example:} {\em The radio works fine on the AM band, but not on the
FM band.}

{\bf What to do:} Eliminate from the list of possible causes, anything
in the radio necessary for the AM band's function. Whatever the source
of the problem is, it is specific to the FM band and not to the AM band.
This eliminates the audio amplifier, speakers, fuse, power supply, and
almost all external wiring. Being able to eliminate sections of the
system as possible failures reduces the scope of the problem and makes
the rest of the troubleshooting procedure more efficient.

\stopsubsection

\startsubsection[title={Hypothesize},reference={sec:xtocid15790976}]

Based on your knowledge of how a system works, think of various kinds of
failures that would cause this problem (or these phenomena) to occur,
and check for those failures (starting with the most likely based on
circumstances, history, or knowledge of component weaknesses).

{\bf Example:} {\em The car's engine is overheating.}

{\bf What to do:} Consider possible causes for overheating, based on
what you know of engine operation. Either the engine is generating too
much heat, or not getting rid of the heat well enough (most likely the
latter). Brainstorm some possible causes: a loose fan belt, clogged
radiator, bad water pump, low coolant level, etc. Investigate each one
of those possibilities before investigating alternatives.

\stopsubsection

\stopsection

\startsection[title={Specific Troubleshooting
Techniques},reference={sec:xtocid15790977}]

After applying some of the general troubleshooting tips to narrow the
scope of a problem's location, there are techniques useful in further
isolating it. Here are a few:

\startsubsection[title={Swap Identical
Components},reference={sec:xtocid15790978}]

In a system with identical or parallel subsystems, swap components
between those subsystems and see whether or not the problem moves with
the swapped component. If it does, you've just swapped the faulty
component; if it doesn't, keep searching!

This is a powerful troubleshooting method, because it gives you both a
positive and a negative indication of the swapped component's fault:
when the bad part is exchanged between identical systems, the formerly
broken subsystem will start working again and the formerly good
subsystem will fail.

I was once able to troubleshoot an elusive problem with an automotive
engine ignition system using this method: I happened to have a friend
with an automobile sharing the exact same model of ignition system. We
swapped parts between the engines (distributor, spark plug wires,
ignition coil --- one at a time) until the problem moved to the other
vehicle. The problem happened to be a \quotation{weak} ignition coil,
and it only manifested itself under heavy load (a condition that could
not be simulated in my garage). Normally, this type of problem could
only be pinpointed using an ignition system analyzer (or oscilloscope)
{\em and} a dynamometer to simulate loaded driving conditions. This
technique, however, confirmed the source of the problem with
100\letterpercent{} accuracy, using no diagnostic equipment whatsoever.

Occasionally you may swap a component and find that the problem still
exists, but has changed in some way. This tells you that the components
you just swapped are {\em somehow different} (different calibration,
different function), and nothing more. However, don't dismiss this
information just because it doesn't lead you straight to the problem --
look for other changes in the system as a whole as a result of the swap,
and try to figure out what these changes tell you about the source of
the problem.

An important caveat to this technique is the possibility of causing
further damage. Suppose a component has failed because of another, less
conspicuous failure in the system. Swapping the failed component with a
good component will cause the good component to fail as well. For
example, suppose that a circuit develops a short, which
\quotation{blows} the protective fuse for that circuit. The blown fuse
is not evident by inspection, and you don't have a meter to electrically
test the fuse, so you decide to swap the suspect fuse with one of the
same rating from a working circuit. As a result of this, the good fuse
that you move to the shorted circuit blows as well, leaving you with two
blown fuses and two non-working circuits. At least you know for certain
that the original fuse {\em was} blown, because the circuit it was moved
to stopped working after the swap, but this knowledge was gained only
through the loss of a good fuse and the additional \quotation{down time}
of the second circuit.

Another example to illustrate this caveat is the ignition system problem
previously mentioned. Suppose that the \quotation{weak} ignition coil
had caused the engine to backfire, damaging the muffler. If swapping
ignition system components with another vehicle causes the problem to
move to the other vehicle, damage may be done to the other vehicle's
muffler as well. As a general rule, the technique of swapping identical
components should be used only when there is minimal chance of causing
additional damage. It is an excellent technique for isolating
non-destructive problems.

{\bf Example 1:} {\em You're working on a CNC machine tool with X, Y,
and Z-axis drives. The Y axis is not working, but the X and Z axes are
working. All three axes share identical components (feedback encoders,
servo motor drives, servo motors).}

{\bf What to do:} Exchange these identical components, one at a time, Y
axis and either one of the working axes (X or Z), and see after each
swap whether or not the problem has moved with the swap.

{\bf Example 2:} {\em A stereo system produces no sound on the left
speaker, but the right speaker works just fine.}

{\bf What to do:} Try swapping respective components between the two
channels and see if the problem changes sides, from left to right. When
it does, you've found the defective component. For instance, you could
swap the speakers between channels: if the problem moves to the other
side (i.e.~the same speaker that was dead before is still dead, now that
its connected to the right channel cable) then you know that speaker is
bad. If the problem stays on the same side (i.e.~the speaker formerly
silent is now producing sound after having been moved to the other side
of the room and connected to the other cable), then you know the
speakers are fine, and the problem must lie somewhere else (perhaps in
the cable connecting the silent speaker to the amplifier, or in the
amplifier itself).

If the speakers have been verified as good, then you could check the
cables using the same method. Swap the cables so that each one now
connects to the other channel of the amplifier and to the other speaker.
Again, if the problem changes sides (i.e.~now the right speaker is now
\quotation{dead} and the left speaker now produces sound), then the
cable now connected to the right speaker must be defective. If neither
swap (the speakers nor the cables) causes the problem to change sides
from left to right, then the problem must lie within the amplifier
(i.e.~the left channel output must be \quotation{dead}).

\stopsubsection

\startsubsection[title={Remove Parallel
Components},reference={sec:xtocid15790979}]

If a system is composed of several parallel or redundant components
which can be removed without crippling the whole system, start removing
these components (one at a time) and see if things start to work again.

{\bf Example 1:} {\em A \quotation{star} topology communications network
between several computers has failed. None of the computers are able to
communicate with each other.}

{\bf What to do:} Try unplugging the computers, one at a time from the
network, and see if the network starts working again after one of them
is unplugged. If it does, then that last unplugged computer may be the
one at fault (it may have been \quotation{jamming} the network by
constantly outputting data or noise).

{\bf Example 2:} {\em A household fuse keeps blowing (or the breaker
keeps tripping open) after a short amount of time.}

{\bf What to do:} Unplug appliances from that circuit until the fuse or
breaker quits interrupting the circuit. If you can eliminate the problem
by unplugging a single appliance, then that appliance might be
defective. If you find that unplugging almost any appliance solves the
problem, then the circuit may simply be overloaded by too many
appliances, neither of them defective.

\stopsubsection

\startsubsection[title={Divide System into Sections and Test Those
Sections},reference={sec:xtocid157909710}]

In a system with multiple sections or stages, carefully measure the
variables going in and out of each stage until you find a stage where
things don't look right.

{\bf Example 1:} {\em A radio is not working (producing no sound at the
speaker))}

{\bf What to do:} Divide the circuitry into stages: tuning stage, mixing
stages, amplifier stage, all the way through to the speaker(s). Measure
signals at test points between these stages and tell whether or not a
stage is working properly.

{\bf Example 2:} {\em An analog summer circuit, shown in Figure~1, is
not functioning properly.}

\startplacefigure[reference=fig:01034,title={Schematic Analog Summer
Circuit}]
{\externalfigure[/home/david/Projects/lessons_eee/Explanations/TroubleShoot/media/01034.png][width=0.5\textwidth]}
\stopplacefigure

{\bf What to do:} I would test the passive averager network (the three
resistors at the lower-left corner of the schematic of Figure~1) to see
that the proper (averaged) voltage was seen at the non-inverting input
of the op-amp. I would then measure the voltage at the inverting input
to see if it was the same as at the non-inverting input (or,
alternatively, measure the voltage difference between the two inputs of
the op-amp, as it should be zero). Continue testing sections of the
circuit (or just test points within the circuit) to see if you measure
the expected voltages and currents.

\stopsubsection

\startsubsection[title={Simplify And
Rebuild},reference={sec:xtocid157909711}]

Closely related to the strategy of dividing a system into sections, this
is actually a design and fabrication technique useful for new circuits,
machines, or systems. It's always easier begin the design and
construction process in little steps, leading to larger and larger
steps, rather than to build the whole thing at once and try to
troubleshoot it as a whole.

Suppose that someone were building a custom automobile. He or she would
be foolish to bolt all the parts together without checking and testing
components and subsystems as they went along, expecting everything to
work perfectly after its all assembled. Ideally, the builder would check
the proper operation of components along the way through the
construction process: start and tune the engine {\em before} its
connected to the drivetrain, check for wiring problems {\em before} all
the cover panels are put in place, check the brake system in the
driveway {\em before} taking it out on the road, etc.

Countless times I've witnessed students build a complex experimental
circuit and have trouble getting it to work because they didn't stop to
check things along the way: test all resistors {\em before} plugging
them into place, make sure the power supply is regulating voltage
adequately {\em before} trying to power anything with it, etc. It is
human nature to rush to completion of a project, thinking that such
checks are a waste of valuable time. However, more time will be wasted
in troubleshooting a malfunctioning circuit than would be spent checking
the operation of subsystems throughout the process of construction.

Take the example of the analog summer circuit in the previous section
for example: what if it wasn't working properly? How would you simplify
it and test it in stages? Well, you could reconnect the op-amp as a
basic comparator and see if its responsive to differential input
voltages, and/or connect it as a voltage follower (buffer) and see if it
outputs the same analog voltage as what is input. If it doesn't perform
these simple functions, it will never perform its function in the summer
circuit! By stripping away the complexity of the summer circuit, paring
it down to an (almost) bare op-amp, you can test that component's
functionality and then build from there (add resistor feedback and check
for voltage amplification, then add input resistors and check for
voltage summing), checking for expected results along the way.

\stopsubsection

\startsubsection[title={Trap a Signal},reference={sec:xtocid157909712}]

Set up instrumentation (such as a datalogger, chart recorder, or
multimeter set on \quotation{record} mode) to monitor a signal over a
period of time. This is especially helpful when tracking down
intermittent problems, which have a way of showing up the moment you've
turned your back and walked away.

This may be essential for proving what happens first in a fast-acting
system. Many fast systems (especially shutdown \quotation{trip} systems)
have a \quotation{first out} monitoring capability to provide this kind
of data.

{\bf Example \#1:} {\em A turbine control system shuts automatically in
response to an abnormal condition. By the time a technician arrives at
the scene to survey the turbine's condition, however, everything is in a
\quotation{down} state and its impossible to tell what signal or
condition was responsible for the initial shutdown, as all operating
parameters are now \quotation{abnormal.}}

{\bf What to do:} One technician I knew used a videocamera to record the
turbine control panel, so he could see what happened (by indications on
the gauges) first in an automatic-shutdown event. Simply by looking at
the panel after the fact, there was no way to tell {\em which} signal
shut the turbine down, but the videotape playback would show what
happened in sequence, down to a frame-by-frame time resolution.

{\bf Example \#2}: {\em An alarm system is falsely triggering, and you
suspect it may be due to a specific wire connection going bad.
Unfortunately, the problem never manifests itself while you're watching
it!}

{\bf What to do:} Many modern digital multimeters are equipped with
\quotation{record} settings, whereby they can monitor a voltage,
current, or resistance over time and note whether that measurement
deviates substantially from a regular value. This is an invaluable tool
for use in \quotation{intermittent} electronic system failures.

\stopsubsection

\stopsection

\startsection[title={Likely Failures in Proven
Systems},reference={sec:xtocid157909713}]

The following problems are arranged in order from most likely to least
likely, top to bottom. This order has been determined largely from
personal experience troubleshooting electrical and electronic problems
in automotive, industry, and home applications. This order also assumes
a circuit or system that has been proven to function as designed and has
failed after substantial operation time. Problems experienced in newly
assembled circuits and systems do not necessarily exhibit the same
probabilities of occurrence.

\startsubsection[title={Operator Error},reference={sec:xtocid157909714}]

A frequent cause of system failure is error on the part of those human
beings operating it. This cause of trouble is placed at the top of the
list, but of course the actual likelihood depends largely on the
particular individuals responsible for operation. When operator error is
the cause of a failure, it is {\em unlikely} that it will be admitted
prior to investigation. I do not mean to suggest that operators are
incompetent and irresponsible --- quite the contrary: these people are
often your best teachers for learning system function and obtaining a
history of failure --- but the reality of human error cannot be
overlooked. A positive attitude coupled with good interpersonal skills
on the part of the troubleshooter goes a long way in troubleshooting
when human error is the root cause of failure.

\stopsubsection

\startsubsection[title={Bad Wire
Connections},reference={sec:xtocid157909715}]

As incredible as this may sound to the new student of electronics, a
high percentage of electrical and electronic system problems are caused
by a very simple source of trouble: poor (i.e.~open or shorted) wire
connections. This is especially true when the environment is hostile,
including such factors as high vibration and/or a corrosive atmosphere.
Connection points found in any variety of plug-and-socket connector,
terminal strip, or splice are at the greatest risk for failure. The
category of \quotation{connections} also includes mechanical switch
contacts, which can be thought of as a high-cycle connector. Improper
wire termination lugs (such as a compression-style connector crimped on
the end of a {\em solid} wire --- a definite {\em faux pas}) can cause
high-resistance connections after a period of trouble-free service.

It should be noted that connections in low-voltage systems tend to be
far more troublesome than connections in high-voltage systems. The main
reason for this is the effect of arcing across a discontinuity (circuit
break) in higher-voltage systems tends to blast away insulating layers
of dirt and corrosion, and may even weld the two ends together if
sustained long enough. Low-voltage systems tend not to generate such
vigorous arcing across the gap of a circuit break, and also tend to be
more sensitive to additional resistance in the circuit. Mechanical
switch contacts used in low-voltage systems benefit from having the
recommended minimum {\em wetting current} conducted through them to
promote a healthy amount of arcing upon opening, even if this level of
current is not necessary for the operation of other circuit components.

Although {\em open} failures tend to more common than {\em shorted}
failures, \quotation{shorts} still constitute a substantial percentage
of wiring failure modes. Many shorts are caused by degradation of wire
insulation. This, again, is especially true when the environment is
hostile, including such factors as high vibration, high heat, high
humidity, or high voltage. It is rare to find a mechanical switch
contact that is failed shorted, except in the case of high-current
contacts where contact \quotation{welding} may occur in overcurrent
conditions. Shorts may also be caused by conductive buildup across
terminal strip sections or the backs of printed circuit boards.

A common case of shorted wiring is the {\em ground fault}, where a
conductor accidently makes contact with either earth or chassis ground.
This may change the voltage(s) present between other conductors in the
circuit and ground, thereby causing bizarre system malfunctions and/or
personnel hazard.

\stopsubsection

\startsubsection[title={Power Supply
Problems},reference={sec:xtocid157909716}]

These generally consist of tripped overcurrent protection devices or
damage due to overheating. Although power supply circuitry is usually
less complex than the circuitry being powered, and therefore should
figure to be less prone to failure on that basis alone, it generally
handles more power than any other portion of the system and therefore
must deal with greater voltages and/or currents. Also, because of its
relative design simplicity, a system's power supply may not receive the
engineering attention it deserves, most of the engineering focus devoted
to more glamorous parts of the system.

\stopsubsection

\startsubsection[title={Active
Components},reference={sec:xtocid157909717}]

Active components (amplification devices) tend to fail with greater
regularity than passive (non-amplifying) devices, due to their greater
complexity and tendency to amplify overvoltage/overcurrent conditions.
Semiconductor devices are notoriously prone to failure due to electrical
transient (voltage/current surge) overloading and thermal (heat)
overloading. Electron tube devices are far more resistant to both of
these failure modes, but are generally more prone to mechanical failures
due to their fragile construction.

\stopsubsection

\startsubsection[title={Passive
Components},reference={sec:xtocid157909718}]

Non-amplifying components are the most rugged of all, their relative
simplicity granting them a statistical advantage over active devices.
The following list gives an approximate relation of failure
probabilities (again, top being the most likely and bottom being the
least likely):

\startitemize[packed]
\item
  Capacitors (shorted), especially {\em electrolytic} capacitors. The
  paste electrolyte tends to lose moisture with age, leading to failure.
  Thin dielectric layers may be punctured by overvoltage transients.
\item
  Diodes open (rectifying diodes) or shorted (Zener diodes).
\item
  Inductor and transformer windings open or shorted to conductive core.
  Failures related to overheating (insulation breakdown) are easily
  detected by smell.
\item
  Resistors open, almost never shorted. Usually this is due to
  overcurrent heating, although it is less frequently caused by
  overvoltage transient (arc-over) or physical damage (vibration or
  impact). Resistors may also change resistance value if overheated!
\stopitemize

\stopsubsection

\stopsection

\startsection[title={Likely Failures in Unproven
Systems},reference={sec:xtocid157909719}]

\startblockquote
{\em \quotation{All men are liable to error;}}

{\bf John Locke}
\stopblockquote

Whereas the last section deals with component failures in systems that
have been successfully operating for some time, this section
concentrates on the problems plaguing brand-new systems. In this case,
failure modes are generally not of the aging kind, but are related to
mistakes in design and assembly caused by human beings.

\startsubsection[title={Wiring
Problems},reference={sec:xtocid157909720}]

In this case, bad connections are usually due to assembly error, such as
connection to the wrong point or poor connector fabrication. Shorted
failures are also seen, but usually involve misconnections (conductors
inadvertently attached to grounding points) or wires pinched under box
covers.

Another wiring-related problem seen in new systems is that of
electrostatic or electromagnetic interference between different circuits
by way of close wiring proximity. This kind of problem is easily created
by routing sets of wires too close to each other (especially routing
signal cables close to power conductors), and tends to be very difficult
to identify and locate with test equipment.

\stopsubsection

\startsubsection[title={Power Supply
Problems},reference={sec:xtocid157909721}]

Blown fuses and tripped circuit breakers are likely sources of trouble,
especially if the project in question is an addition to an
already-functioning system. Loads may be larger than expected, resulting
in overloading and subsequent failure of power supplies.

\stopsubsection

\startsubsection[title={Defective
Components},reference={sec:xtocid157909722}]

In the case of a newly-assembled system, component fault probabilities
are not as predictable as in the case of an operating system that fails
with age. {\em Any} type of component --- active or passive --- may be
found defective or of imprecise value \quotation{out of the box} with
roughly equal probability, barring any specific sensitivities in
shipping (i.e fragile vacuum tubes or electrostatically sensitive
semiconductor components). Moreover, these types of failures are not
always as easy to identify by sight or smell as an age- or
transient-induced failure.

\stopsubsection

\startsubsection[title={Improper System
Configuration},reference={sec:xtocid157909723}]

Increasingly seen in large systems using microprocessor-based
components, \quotation{programming} issues can still plague
non-microprocessor systems in the form of incorrect time-delay relay
settings, limit switch calibrations, and drum switch sequences. Complex
components having configuration \quotation{jumpers} or switches to
control behavior may not be \quotation{programmed} properly.

Components may be used in a new system outside of their tolerable
ranges. Resistors, for example, with too low of power ratings, of too
great of tolerance, may have been installed. Sensors, instruments, and
controlling mechanisms may be uncalibrated, or calibrated to the wrong
ranges.

\stopsubsection

\startsubsection[title={Design Error},reference={sec:xtocid157909724}]

Perhaps the most difficult to pinpoint and the slowest to be recognized
(especially by the chief designer) is the problem of design error, where
the system fails to function simply because it {\em cannot} function as
designed. This may be as trivial as the designer specifying the wrong
components in a system, or as fundamental as a system not working due to
the designer's improper knowledge of physics.

I once saw a turbine control system installed that used a low-pressure
switch on the lubrication oil tubing to shut down the turbine if oil
pressure dropped to an insufficient level. The oil pressure for
lubrication was supplied by an oil pump turned by the turbine. When
installed, the turbine refused to start. Why? Because when it was
stopped, the oil pump was not turning, thus there was no oil pressure to
lubricate the turbine. The low-oil-pressure switch detected this
condition and the control system maintained the turbine in shutdown
mode, preventing it from starting. This is a classic example of a design
flaw, and it could only be corrected by a change in the system logic.

While most design flaws manifest themselves early in the operational
life of the system, some remain hidden until just the right conditions
exist to trigger the fault. These types of flaws are the most difficult
to uncover, as the troubleshooter usually overlooks the possibility of
design error due to the fact that the system is assumed to be
\quotation{proven.} The example of the turbine lubrication system was a
design flaw impossible to ignore on start-up. An example of a
\quotation{hidden} design flaw might be a faulty emergency coolant
system for a machine, designed to remain inactive until certain abnormal
conditions are reached --- conditions which might never be experienced
in the life of the system.

\stopsubsection

\stopsection

\startsection[title={Potential
Pitfalls},reference={sec:xtocid157909725}]

Fallacious reasoning and poor interpersonal relations account for more
failed or belabored troubleshooting efforts than any other impediments.
With this in mind, the aspiring troubleshooter needs to be familiar with
a few common troubleshooting mistakes.

{\bf Trusting that a brand-new component will always be good.} While it
is generally true that a new component will be in good condition, it is
not {\em always} true. It is also possible that a component has been
mis-labeled and may have the wrong value (usually this mis-labeling is a
mistake made at the point of distribution or warehousing and not at the
manufacturer, but again, {\em not always}!).

{\bf Not periodically checking your test equipment.} This is especially
true with battery-powered meters, as weak batteries may give spurious
readings. When using meters to safety-check for dangerous voltage,
remember to test the meter on a known source of voltage both {\em
before} and {\em after} checking the circuit to be serviced, to make
sure the meter is in proper operating condition.

{\bf Assuming there is only one failure to account for the problem.}
Single-failure system problems are ideal for troubleshooting, but
sometimes failures come in multiple numbers. In some instances, the
failure of one component may lead to a system condition that damages
other components. Sometimes a component in marginal condition goes
undetected for a long time, then when another component fails the system
suffers from problems with {\em both} components.

{\bf Mistaking coincidence for causality.} Just because two events
occurred at nearly the same time does {\em not} necessarily mean one
event {\em caused} the other! They may be both consequences of a common
cause, or they may be totally unrelated! If possible, try to duplicate
the same condition suspected to be the cause and see if the event
suspected to be the coincidence happens again. If not, then there is
either no causal relationship as assumed. This may mean there is no
causal relationship between the two events whatsoever, or that there is
a causal relationship, but just not the one you expected.

{\bf Self-induced blindness.} After a long effort at troubleshooting a
difficult problem, you may become tired and begin to overlook crucial
clues to the problem. Take a break and let someone else look at it for a
while. You will be amazed at what a difference this can make. On the
other hand, it is generally a bad idea to solicit help at the start of
the troubleshooting process. Effective troubleshooting involves complex,
multi-level thinking, which is not easily communicated with others. More
often than not, \quotation{team troubleshooting} takes more time and
causes more frustration than doing it yourself. An exception to this
rule is when the knowledge of the troubleshooters is complementary: for
example, a technician who knows electronics but not machine operation,
teamed with an operator who knows machine function but not electronics.

{\bf Failing to question the troubleshooting work of others on the same
job.} This may sound rather cynical and misanthropic, but it is sound
scientific practice. Because it is easy to overlook important details,
troubleshooting data received from another troubleshooter should be
personally verified before proceeding. This is a common situation when
troubleshooters \quotation{change shifts} and a technician takes over
for another technician who is leaving before the job is done. It is
important to exchange information, but do not assume the prior
technician checked everything they said they did, or checked it
perfectly. I've been hindered in my troubleshooting efforts on many
occasions by failing to verify what someone else told me they checked.

{\bf Being pressured to \quotation{hurry up.}} When an important system
fails, there will be pressure from other people to fix the problem as
quickly as possible. As they say in business, \quotation{time is money.}
Having been on the receiving end of this pressure many times, I can
understand the need for expedience. However, in many cases there is a
higher priority: caution. If the system in question harbors great danger
to life and limb, the pressure to \quotation{hurry up} may result in
injury or death. At the very least, hasty repairs may result in further
damage when the system is restarted. Most failures can be recovered or
at least temporarily repaired in short time if approached intelligently.
Improper \quotation{fixes} resulting in haste often lead to damage that
{\em cannot} be recovered in short time, if ever. If the potential for
greater harm is present, the troubleshooter needs to politely address
the pressure received from others, and maintain their perspective in the
midst of chaos. Interpersonal skills are just as important in this realm
as technical ability!

{\bf Finger-pointing.} It is all too easy to blame a problem on someone
else, for reasons of ignorance, pride, laziness, or some other
unfortunate facet of human nature. When the responsibility for system
maintenance is divided into departments or work crews, troubleshooting
efforts are often hindered by blame cast between groups. \quotation{It's
a mechanical problem \ldots{} its an electrical problem \ldots{} its an
instrument problem \ldots{}} ad infinitum, ad nauseum, is all too common
in the workplace. I have found that a positive attitude does more to
quench the fires of blame than anything else.

On one particular job, I was summoned to fix a problem in a hydraulic
system assumed to be related to the electronic metering and controls. My
troubleshooting isolated the source of trouble to a faulty control
valve, which was the domain of the millwright (mechanical) crew. I knew
that the millwright on shift was a contentious person, so I expected
trouble if I simply passed the problem on to his department. Instead, I
politely explained to him and his supervisor the nature of the problem
as well as a brief synopsis of my reasoning, then proceeded to help him
replace the faulty valve, even though it wasn't \quotation{my}
responsibility to do so. As a result, the problem was fixed very
quickly, and I gained the respect of the millwright.

\stopsection

\startsection[title={Contributors},reference={sec:xtocid157909726}]

Contributors to this chapter are listed in chronological order of their
contributions, from most recent to first. See Appendix 2 (Contributor
List) for dates and contact information.

{\bf Alejandro Gamero Divasto} (January 2002): contributed
troubleshooting tips regarding potential hazards of swapping two similar
components, avoiding pressure placed on the troubleshooter, perils of
\quotation{team} troubleshooting, wisdom of recording system history,
operator error as a cause of failure, and the perils of finger-pointing.

\stopsection

\stopchapter

